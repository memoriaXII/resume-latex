%=== 通用設定與套件 ===
\documentclass[11pt,a4paper,sans]{moderncv}
\moderncvstyle{classic}    % 可選:classic, casual, banking, oldstyle, fancy
\moderncvcolor{blue}       % 可選:blue, black, burgundy, green, grey, orange, purple, red
\usepackage[scale=0.75]{geometry}

% 必要字型設定
\ifxetexorluatex
  \usepackage{fontspec}
  \defaultfontfeatures{Ligatures=TeX}
  \setmainfont{Latin Modern Roman}
  \setsansfont{Latin Modern Sans}
\else
  \usepackage[T1]{fontenc}
  \usepackage{lmodern}
\fi

%=== 個人資料 ===
\name{Cheng Lei}{Chou (Eli)}
\title{Curriculum Vitae}  % 或職稱
\address{Taipei, Taiwan}{}{}
\phone[mobile]{+886-912-345-678}
\email{eli@example.com}
\homepage{memoriaXII.github.io}{memoriaXII.github.io}

%=== 正文開始 ===
\begin{document}

%--- Resume (CV) 部分 ---
\makecvtitle
\section{Education}
\cventry{2020--2024}{Bachelor of Science in Computer Science}{National Taiwan University}{Taipei}{}{GPA: 4.0/4.3}
\section{Experience}
\cventry{2024--Present}{Frontend Developer Intern}{Awesome Tech Co.}{Taipei}{}{Developed Angular-based dashboards with REST APIs.}
% (你可以按照格式替換成你自己的內容)

\clearpage

%--- Cover Letter 部分 ---
\recipient{Carissa Yates}{Hiring Manager\\ShellShock}
\date{\today}
\opening{Dear Carissa Yates,}
\closing{Sincerely,}
\makelettertitle

% 若希望內容左右對齊,可加入下列命令以取消預設的 ragged-right 寫法 :contentReference[oaicite:2]{index=2}
\usepackage{ragged2e}
\justifying

I am writing to express my strong interest in the Angular Frontend Developer position at ShellShock. With my experience in building scalable and user-centric Angular applications and my dedication to creating seamless UIs, I am confident I am a great fit for this role.

During my time at Awesome Tech Co., I led development of performant, accessible components that boosted user satisfaction by 30%. I thrive in cross-functional teams and highly value innovation—qualities aligned with ShellShock’s vision.

I look forward to the opportunity to discuss how I can contribute to your team.

\makeletterclosing

\end{document}
